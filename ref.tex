\chapter{References}
\markboth{REFERENCES}{REFERENCES}
\begin{description}
\setlength{\itemsep}{0pt}
%\setlength{\parsep}{0pt}
%\setlength{\parindent}{0pt}
%\setlength{\parskip}{0pt}
\footnotesize

\item [Alley]R.B. 2000. Ice-core evidence of abrupt climate change.
\textit{Proceedings of National Academy of Sciences} 97 (4): 1331--1334.

\item[Apel]J.R. 1987. \textit{Principles of Ocean Physics}.  New York:
Academic Press.

\item [Anderson]J.D. 2005. Ludwig Prandtl's Boundary
Layer. \textit{Physics Today} 58 (12): 42--48.

\item [Andersen]O.B., P.L. Woodworth, and
R.A. Flather. 1995. Intercomparison of recent ocean tide
models. \textit{J. Geophysical Research} 100 (C12): 25,262--25,282.

\item [Arthur]R.S. 1960. A review of the calculation of ocean currents
at the equator. \textit{Deep-Sea Research} 6 (4): 287--297.

\item [Atlas]R., R.N. Hoffman, and S.C. Bloom. 1993. Surface wind
velocity over the oceans. In: \textit{Atlas of satellite
observations related to global change}. Edited by R. J. Gurney,
J. L. Foster and C. L. Parkinson. 129--140. Cambridge: University Press.

\item [Baker]D.J. 1981. Ocean instruments and experiment design. In
\textit{Evolution of Physical Oceanography: Scientific Surveys in
Honor of Henry Stommel}. Edited by B. A. Warren and
C. Wunsch. 396--433. Cambridge: Massachusetts Institute of Technology Press.

\item [Barnett]T.P., M. Latif, N.E. Graham, M. Flugel, S. Pazan, and
W. White.  1993. \textsc{enso} and \textsc{enso}-related
predictability. Part I: Prediction of equatorial Pacific sea surface
temperature in a hybrid coupled ocean-atmosphere
model. \textit{Journal of Climate} 6: 1545--1566.

\item [Barnier]B., L. Siefridt, and P. Marchesiello. 1995. Thermal
forcing for a global ocean circulation model using a three-year
climatology of \textsc{ecmwf} analyses. \textit{Journal of Marine
Systems} 6: 393--380.

\item [Barnston] A.G., Y. Hea, and M.H. Glantz. 1999. Predictive Skill
of Statistical and Dynamical Climate Models in SST Forecasts during
the 1997--98 El Ni\~{n}o Episode and the 1998 La Ni\~{n}a
Onset. \textit{Bulletin of the American Meteorological Society} 80 (2):
217--243.

\item [Batchelor]G.K. 1967. \textit{An Introduction to Fluid Dynamics}.
Cambridge: University Press.

\item [Baumgartner]A., and E. Reichel. 1975. \textit{The World Water Balance}. 
New York: Elsevier.

\item[Beardsley]R.C., A.G. Enriquez, C.A. Friehe, and
C.A. Alessi. 1997.  Intercomparison of aircraft and buoy measurements
of wind speed and wind stress during \textsc{smile}. \textit{Journal
of Atmospheric and Oceanic Technology} 14: 969--977.

\item [Behringer]D.W., M. Ji and A. Leetmaa. 1998. An improved coupled
  model for \textsc{enso} prediction and implications for ocean
  initialization: Part 1: The ocean data assimilation. \textit{Monthly
    Weather Review} 126(4): 1013--1021.

\item [Bennett]A.F.  1992. \textit{Inverse Methods in Physical
  Oceanography}.  Cambridge: University Press.

\item [Berlinski]D. 1996. The end of materialist
  science. \textit{Forbes ASAP} (December 2, 1996): 146--160.

\item [Binder]R.C. 1949. \textit{Fluid Mechanics}. Second ed. New
  York: Prentice--Hall.

\item [Bjerknes]J. 1966. The possible response of the atmospheric
  Hadley circulation to equatorial anomalies of ocean
  temperature. \textit{Tellus} 4: 820-929.

\item [Bjerknes]J. 1972. Large-scale atmospheric response to the
  1964--65 Pacific equatorial warming. \textit{Journal of Physical
    Oceanography} 2 (3): 212--217.

\item [Bjerknes]V. and J.W. Sandstr\"{o}m. 1910. Dynamic Meteorology
  and Hydrography, Part I. Statics. Carnegie Institution of Washington
  DC, Publication No. 88.

\item [Bleck]R. 2002. An oceanic general circulation model framed in
  hybrid isopycnic-Cartesian coordinates. \textit{Ocean Modeling} 4:
  55--88.

\item [Blumberg]A.F., and G.L. Mellor. 1987. A description of a
  three-dimensional ocean circulation model. In:
  \textit{Three-Dimensional Coastal Ocean Models}. Edited by
  N. S. Heaps. 1--16. Washington, DC: American Geophysical Union.

\item [Bond]G., H. Heinrich, W. Broecker, L. Labeyrie, J. McManus,
  J. Andrews, S. Huon, R. Jantschik, S. Clasen, C. Simet, K. Tedesco,
  M. Klas, G. Bonani, and S. Ivy. 1992. Evidence for massive
  discharges of icebergs into the North Atlantic ocean during the last
  glacial period. \textit{Nature} 360, 245--.

\item [Boville]B.A., and P.R. Gent. 1998. The \textsc{ncar} Climate
  System Model, Version One. \textit{Journal of Climate} 11 (6):
  1115--1130.

\item[Bowden]K.F. 1962. Turbulence. In: \textit{The Sea Volume
  1}. Edited by M. N.  Hill. 802--825. New York: Interscience
  Publishers, John Wiley and Sons.

\item [Bracewell]R.N. 1986. \textit{The Fourier Transform and Its
  Applications}.  Second, revised ed.  New York: McGraw-Hill
  Publishing Company.

\item [Brauer]A., G. H. Haug, et al. 2008. An abrupt wind shift in
  western Europe at the onset of the Younger Dryas cold
  period. \textit{Nature Geoscience} 1 (8): 520--523.
	
\item [Broecker]W.S. 1987. Unpleasant surprises in the greenhouse?
  \textit{Nature} 328: 123--126.

\item [Broecker]W.S. 1997. Thermohaline circulation, the Achilles heel
  of our climate system: Will man-made CO$_2$ upset the current
  balance? \textit{Science} 278 (5343): 1582--1588.

\item [Bryan] K. 1969. A numerical method for the study of the world
  ocean.  \textit{Journal of Computational Physics} 4: 347--376.

\item [Bryden]H.L. and T.H. Kinder. 1991. Steady two-layer exchange
  through the Strait of Gibraltar. \textit{Deep Sea Research} 38
  Supplement 1A: S445--S463.
	
\item [Carritt]D.E., and J.H. Carpenter. 1959. The composition of sea
  water and the salinity-chlorinity-density problems. National Academy
  of Sciences-National Research Council, Report 600: 67--86.

\item [Cartwright]D.E. 1999. \textit{Tides: A Scientific
  History}. Cambridge, University Press.
	
\item [Cazenave]A., and J.Y. Royer. 2001. Applications to marine
  Geophysics. In \textit{Satellite Altimetry and earth
    Sciences}. 407--439. San Diego: Academic Press.

\item [Cess]R.D., M.H. Zhang, P.Minnis, L.Corsetti, E.G. Dutton,
  B.W. Forgan, D.P. Garber, W.L. Gates, J.J. Hack, E.F. Harrison,
  X. Jing, J.T. Kiehl, C.N.  Long, J.-J. Morcrette, G.L. Potter,
  V. Ramanathan, B. Subasilar, C.H. Whitlock, D.F. Young, and
  Y. Zhou. 1995. Absorption of solar radiation by clouds: Observations
  versus models. \textit{Science} 267 (5197): 496--499.

\item[Chambers]D.P., B.D. Tapley, and Stewart, R.H. 1998. Measuring
  heat storage changes in the equatorial Pacific: A comparison between
  \textsc{Topex} altimetry and Tropical Atmo\-sphere-Ocean
  buoys. \textit{Journal of Geophysical Research} 103(C9):
  18,591--18,597.

\item [Charnock]H. 1955. Wind stress on a water
  surface. \textit{Quarterly Journal Royal Meteorological Society} 81:
  639--640.

\item [Chelton]D.B., J.C. Ries, B.J. Haines, L.L. Fu, and
  P.S. Callahan. 2001. Satellite Altimetry. In: \textit{Satellite
    Altimetry and Earth Sciences: A handbook of techniques and
    applications}. Editors: L.-L. Fu and A. Cazenave. Academic Press:
  1--131.
	
\item [Chen]D., S.E. Zebiak, A.J. Busalacchi, and M.A. Cane. 1995. An
  improved procedure for El Ni\~{n}o forecasting: Implications for
  predictability.  \textit{Science} 269 (5231): 1699--1702.

\item [Chereskin]T.K., and D. Roemmich. 1991. A comparison of measured
  and wind-derived Ekman transport at 11\degrees N in the Atlantic
  Ocean.  \textit{Journal of Physical Oceanography} 21 (6): 869--878.

\item [Chen]D., S.E. Zebiak, A.J. Busalacchi, and M.A. Cane. 1995. An
  improved procedure for El Ni\~{n}o forecasting: Implications for
  predictability.  \textit{Science} 269 (5231): 1699--1702.

\item [Chou]S.-H., E. Nelkin, et al. 2004. A comparison of latent heat
  fluxes over global oceans for four flux products. \textit{Journal of
    Climate} 17 (20): 3973--3989.

\item [Church]J. A. 2007. Oceans: A Change in Circulation?
  \textit{Science} 317 (5840): 908--909.
	
\item [Clarke]G.L., G.C. Ewing, and C.J. Lorenzen. 1970. Spectra of
  backscattered light from the sea obtained from aircraft as a measure
  of chlorophyll concentration. \textit{Science} 167: 1119--1121.

\item [Coakley]J.A., and F.P. Bretherton. 1982. Cloud cover from high
  resolution scanner data: Detecting and allowing for partially filled
  fields of view. \textit{Journal of Geophysical. Research} 87 (C7):
  4917--4932.

\item [Cooley]J.W., P.A. Lewis, and P.D. Welch. 1970. The fast Fourier
  transform algorithm: Programming considerations in the calculation
  of sine, cosine and Laplace transforms. \textit{Journal of Sound and
    Vibration} 12: 315--337.

\item [Couper]A. Editor. 1983. \textit{The Times Atlas of the
  Oceans}. New York: Van Nostrand Reinhold Company.

\item [Cox]M.D. 1975. A baroclinic model of the world ocean:
  Preliminary results. In: \textit{Numerical Models of Ocean
    Circulation}: 107--120. Washington, DC: National Academy of
  Sciences.

\item [Cromwell]T., R.B. Montgomery, and E.D. Stroup. 1954. Equatorial
  undercurrent in Pacific Ocean revealed by new
  methods. \textit{Science} 119 (3097): 648--649.

\item[Cushman-Roisin]B. 1994. \textit{Introduction to Geophysical
  Fluid Dynamics}.  Englewood Cliffs: Prentice Hall.

\item[Cutchin]D.L., and R.L. Smith. 1973. Continental shelf waves:
  Low-frequency variations in sea level and currents over the Oregon
  continental shelf. \textit{Journal of Physical Oceanography} 3 (1):
  73--82.

\item [Daley]R. 1991. \textit{Atmospheric Data Analysis}.  Cambridge:
  University Press.

\item [Danabasoglu]G., J.C. McWilliams, and P.R. Gent. 1994. The role
  of mesoscale tracer transports in the global ocean
  circulation. \textit{Science} 264 (5162): 1123--1126.

\item [Dansgaard]W., S.J. Johnsen, H.B. Clausen, D. Dahl-Johnsen, N.
  Gunderstrup, C.U. Hammer, C. Hvidberg, J. Steffensen,
  A. Sveinbjörnsobttir, J.  Jouze and G. Bond. 1993. Evidence for
  general instability of past climate from a 250-kyr ice core
  record. \textit{Nature} 364: 218-220.

\item [Darnell]W.L., F. Staylor, S.K. Gupta, and
  F,M. Denn. 1988. Estimation of surface insolation using
  sun-synchronous satellite data. \textit{Journal of Climate} 1 (8):
  820--835.

\item [Darnell]W.L., W.F. Staylor, S.K. Gupta, N.A. Ritchey, and
  A.C. Wilbur.  1992. Seasonal variation of surface radiation budget
  derived from International Satellite Cloud Climatology Project C1
  data. \textit{Journal of Geophysical Research} 97: 15,741--15,760.

\item [Darwin]Sir G.H. 1911. \textit{The Tides and Kindred Phenomena
  in the Solar System}. 3rd ed.  London: John Murray.

\item [DaSilva]A., C.C. Young, and S. Levitus. 1995. Atlas of surface
  marine data 1994. Vol. 1: Algorithms and procedures. National
  Oceanic and Atmospheric Administration Report.

\item [Davis]R.A. 1987. \textit{Oceanography: An Introduction to the
  Marine Environment}. Dubuque: Wm. C. Brown Publishers.

\item [Davis]R.E., R. DeSzoeke, and P. Niiler. 1981. Variability in
  the upper ocean during \textsc{mile}. Part II: Modeling the mixed
  layer response. \textit{Deep-Sea Research} 28A (12): 1453--1475.

\item [Davis]R.E., D.C. Webb, L.A. Regier, and J. Dufour. 1992. The
  Autonomous Lagrangian Circulation Explorer
  (\textsc{alace}). \textit{Journal of Atmospheric and Oceanic
    Technology} 9: 264--285.

\item [Defant]A. 1961. \textit{Physical Oceanography}. New York:
  Macmillan Company.

\item [Dietrich]G., K. Kalle, W. Krauss, and
  G. Siedler. 1980. \textit{General Oceanography}. 2nd ed. Translated
  by Susanne and Hans Ulrich Roll. New York: John Wiley and Sons
  (Wiley-Interscience).

\item [Dittmar]W. 1884. Report on researches into the composition of
  ocean water, collected by the HMS Challenger. \textit{Challenger
    Reports, Physics and Chemistry} 1.

\item [Dobson]G.M.B. 1914. Pilot balloon ascents at the central flying
  school, Upavon, during the year 1913. \textit{Quarterly Journal of
    the Royal Meteorological Society} 40: 123--135.

\item [Domingues]C. M., J. A. Church, et al 2008. Improved estimates
  of upper-ocean warming and multi-decadal sea-level
  rise. \textit{Nature} 453 (7198): 1090--1093.
	
\item [Doodson]A.T. 1922. Harmonic development of the tide-generating
  potential. \textit{Proceedings of the Royal Society of London} A
  100: 305--329.

\item [Dorman]C.E. and R.H. Bourke. 1981. Precipitation over the
  Atlantic ocean, 30\degrees S to 70\degrees N. \textit{Monthly
    Weather Review} 109: 554--563.

\item [Dritschel]D.G., M. de la T. Juarez and M.H.P. Ambaum. 1999. The
  three-dimensional vortical nature of atmospheric and oceanic
  turbulent flows.  \textit{Physics of Fluids} 11(6): 1512--1520.

\item [Dushaw]B.D., P.F. Worcester, B.D. Cornuelle, and
  B.M. Howe. 1993. On equations for the speed of sound in sea water.
  \textit{Journal of the Acoustical Society of America} 93: 255--275.

\item [Ebbesmeyer]C.C., and W.J. Ingraham. 1992. Shoe spill in the
  North Pacific. \textit{EOS, Transactions of the American Geophysical
    Union} 73 (34): 361, 365.

\item [Ebbesmeyer]C.C., and W.J. Ingraham. 1994. Pacific toy spill
  fuels ocean current pathways research. \textit{EOS Transactions of
    the American Geophysical Union} 75 (37): 425, 427, 430.

\item [Eden]C., and J. Willebrand. 1999. Neutral density revisited.
  \textit{Deep-Sea Research} Part II: Topical Studies in
  Oceanography. 46: 33--54.

\item [Egbert]G.B. and R.D. Ray 2000. Significant dissipation of tidal
  energy in the deep ocean inferred from satellit altimeter
  data. \textit{Nature} 405: 775--778.

\item [Ekman]V.W. 1905. On the influence of the Earth's rotation on
  ocean currents. \textit{Arkiv for Matematik, Astronomi, och Fysik}:
  2 (11).

\item [Emery]W., and P. Schussel. 1989. Global difference between skin
  and bulk sea surface temperatures. \textit{EOS: Transactions of the
    American Geophysical Union} 70 (14): 210--211.

\item [Feynman]R.P., R.B. Leighton, and M. Sands. 1964. \textit{The
  Feynman Lectures on Physics}.  Addison-Wesley Publishing Company.

\item [Fofonoff]N.P., and R.C. Millard. 1983. Algorithms for
  computation of fundamental properties of sea water. \textsc{unesco}
  Technical Papers in Marine Science 44.

\item [Friedman]R.M. 1989. \textit{Appropriating the Weather. Vilhelm
  Bjerknes and the Construction of a Modern Meteorology.}  Ithaca and
  London: Cornell University Press.

\item [Friedrichs]M.A.M., and M.M. Hall. 1993. Deep circulation in the
  tropical North Atlantic. \textit{Journal of Marine Research} 51 (4):
  697--736.

\item[Freilich]M.H., and R.S. Dunbar. 1999. The accuracy of the NSCAT
  1 vector winds: Comparisons with National Data Buoy Center
  buoys. \textit{Journal of Geophysical Research} submitted

\item [Garabato]A.C.N., K.L. Polzin, B.A. King, K.J. Heywood, and
  M. Visbeck. 2004. Widespread Intense Turbulent Mixing in the
  Southern Ocean. \textit{Science} 303 (5655): 210--213.

\item [Garabato]A.C.N., D.P. Stevens, A.J. Watson, and
  W. Roether. 2007. Short-circuiting of the overturning circulation in
  the Antarctic Circumpolar Current. \textit{Nature} 447 (7141):
  194--197.
	
\item[Gargett]A.E., and G. Holloway. 1992. Sensitivity of the
  \textsc{gfdl} ocean model to different diffusivities of heat and
  salt. \textit{Journal of Physical Oceanography} 22 (10): 1158--1177.

\item [Garrett]C. 2006. Turbulent dispersion in the
  ocean. \textit{Progress in Oceanography} 70 (2--4): 113--125.

\item[Gates]W.L., A. Henderson-Sellers, G.J. Boer, C.K. Folland,
  A. Kitoh, B.J. McAvaney, F. Semazzi, N. Smith, A.J. Weaver, and
  Q.-C. Zeng. 1996. \textit{Climate Models--Evaluation}. In:
  \textit{Climate Change 1995}. Edited by J.T. Houghton, L.G.M. Filho,
  B.A. Callander, N. Harris, A. Kattenberg and K.
  Maskell. 229--284. Cambridge: University Press.

\item [Gates]W.L. 1992. \textsc{amip}: The Atmospheric Model
  Intercomparison Project.  \textit{Bulletin American Meteorological
    Society} 73: 1962--1970.

\item[Gent]P.R., and J.C. McWilliams. 1990. Isopycnal mixing in ocean
  circulation models. \textit{Journal of Physical Oceanography} 20:
  150--155.

\item [Gleckler]P.J., and B. Weare. 1997. Uncertainties in global
  ocean surface heat flux climatologies derived from ship
  observance. \textit{Journal of Climate} 10: 2763--2781.

\item [Gill]A.E. 1982. \textit{Atmosphere-Ocean Dynamics}. New York:
  Academic Press.

\item [Gnadadesikan]A. 1999. A simple predictive model for the
  structure of the oceanic pycnocline. \textit{Science} 283 (5410):
  2077--2079.

\item [Goldenberg]S.B., and J.J. O'Brien. 1981. \textit{Monthly
  Weather Review} 109: 1190.

\item [Goldstein]S. 1965. \textit{Modern Developments in Fluid
  Dynamics}: Two Volumes. New York: Dover Publications.

\item [Gordon]H.R., D.K. Clark, J.W. Brown, O.B. Brown, R.H. Evans,
  and W.W.  Broenkow. 1983. Phytoplankton pigment concentrations in
  the Middle Atlantic Bight: comparison of ship determinations and
  \textsc{czcs} estimates. \textit{Applied Physics} 22 (1): 20-36.

\item [Gouretski]V., and K. Jancke. 1995. A consistent
  pre-\textsc{woce} hydrographic data set for the south Atlantic:
  Station data and gridded fields.  \textsc{woce} Report No. 127/95.

\item [Graber]H.C., V.J. Cardone, R.E. Jensen, D.N. Slinn, S.C. Hagen,
  A.T. Cox, M.D. Powell, and C. Grassl. 2006. Coastal Forecasts and
  Storm Surge Predictions for Tropical Cyclones: A Timely Partnership
  Program. \textit{Oceanography} 19 (1): 131--141.
	
\item [Grassl]H. 2000. Status and improvements of coupled general
  circulation models. \textit{Science} 288 (5473): 1991--1997.

\item[Gregg]M.C. 1987. Diapycnal mixing in the thermocline: A review.
  \textit{Journal of Geophysical Research} 92 (C5): 5,249--5,289.

\item [Gregg]M.C. 1991. The study of mixing in the ocean: a brief
  history.  \textit{Oceanography} 4 (1): 39--45.

\item [Hackett]B., L.P. Roed, B. Gjevik, E.A. Martinsen, and
  L.I. Eide. 1995. A review of the Metocean Modeling Project
  (\textsc{mompop}) Part 2: Model validation study. In:
  \textit{Quantitative Skill Assessment for Coastal Ocean
    Models}. Edited by D. R. Lynch and
  A. M. Davies. 307--327. Washington DC: American Geophysical Union.

\item [Haidvogel]D.B., and A. Beckmann. 1998. Numerical models of the
  coastal ocean. In: \textit{The Sea, Volume 10}. Edited by
  K. H. Brink and A. R. Robinson. 457--482. New York: John Wiley and
  Sons.

\item [Haidvogel]D.B. and A. Beckmann. 1999. \textit{Numerical Ocean
  Circulation Modeling}. London, Imperial College Press.

\item [Hall]M.M., and H.L. Bryden. 1982. Direct estimates and
  mechanisms of ocean heat transport. \textit{Deep-Sea Research} 29:
  339--359.

\item [Harrison]D.E. 1989. On climatological monthly mean wind stress
  and wind stress curl fields over the world ocean. \textit{Journal of
    Climate} 2: 57.

\item [Harrison]D.E., and N.K. Larkin. 1996. The \textsc{coads} sea
  level pressure signal: A near-global el Ni\~{n}o composite and time
  series view, 1946--1993.  \textit{Journal of Climate} 9 (12):
  3025--3055.

\item [Harrison]D.E. and N.K. Larkin 1998. El Ni\~{n}o-Southern
  Oscillation sea surface temperature and wind anomalies,
  1946--1993. \textit{Reviews of Geophysics} 36 (3): 353--399.

\item [Hartmann]D.L. 1994. \textit{Global Physical Climatology}.
  Academic Press.

\item[Hasselmann]K. 1961. On the non-linear energy transfer in a
  gravity-wave spectrum Part 1. General theory. \textit{Journal of
    Fluid Mechanics} 12 (4): 481--500.

\item[Hasselmann]K. 1963a. On the non-linear energy transfer in a
  gravity wave spectrum. Part 2. Conservation theorems; wave-particle
  analogy; irreversibility.  \textit{Journal of Fluid Mechanics} 15
  (2): 273--281.

\item[Hasselmann]K. 1963b. On the non-linear energy transfer in a
  gravity wave spectrum. Part 3. Evaluation of the energy flux and
  swell-sea interaction for a Neumann spectrum. \textit{Journal of
    Fluid Mechanics}. 15 (3): 385--398.

\item[Hasselmann]K. 1966. Feynman diagrams and interaction rules of
  wave-wave scattering processes. \textit{Reviews of Geophysical}. 4
  (1): 1--32.

\item [Hasselmann]K. 1970. Wind--driven inertial
  oscillations. \textit{Geophysical Fluid Dynamics} 1: 1 463--502.

\item [Hasselmann]K., T.P. Barnett, E. Bouws, H. Carlson,
  D.E. Cartwright, K.  Enke, J.A. Ewing, H. Gienapp, D.E. Hasselmann,
  P. Kruseman, A. Meerburg, P.  Müller, D.J. Olbers, K. Richter,
  W. Sell, and H. Walden. 1973. Measurements of wind-wave growth and
  swell decay during the Joint North Sea Wave Project
  (\textsc{jonswap}). \textit{Ergänzungsheft zur Deutschen
    Hydrographischen Zeitschrift Reihe} A(8°) (Nr. 12): 95.

\item [Hasselmann]K., and S. Hasselmann. 1991. On the nonlinear
  mapping of an ocean wave spectrum into a synthetic aperture radar
  image spectrum and its inversion. \textit{Journal of Geophysical
    Research} C96 10,713--10,729.

\item [Heaps]N.S., Editor. 1987. \textit{Three-Dimensional Coastal
  Ocean Models}.  Washington DC: American Geophysical Union.

\item [Hinze]J.O. 1975. \textit{Turbulence}. 2nd ed. New York:
  McGraw-Hill.

\item [Hirst]A.C., S.P. O'Farrell, and H.B. Gordon. 2000. Comparison
  of a coupled ocean-atmosphere model with and without oceanic
  eddy-induced advection.  Part I: Oceanic spinup and control
  integration. \textit{Journal of Climate} 13 (1): 139--163.

\item [Hoffman]D., and O.J. Karst. 1975. The theory of the Rayleigh
  distribution and some of its applications. \textit{Journal of Ship
    Research} 19 (3): 172--191.

\item [Hogg]N., J. McWilliams, P. Niiler and J. Price. 2001. Objective
  8---To determine the important processes and balances for the
  dynamics of the general circulation. In: \textit{2001 U.S. WOCE
    Report}. College Station, Texas, U.S. \textsc{woce} Office:
  50--59.

\item [Holloway]G. 1986. Eddies, waves, circulation, and mixing:
  statistical geofluid mechanics. \textit{Annual Reviews of Fluid
    Mechanics} 18: 91--147.

\item [Holloway]G. 1986. Estimation of oceanic eddy transports from
  satellite altimetry. \textit{Nature} 323 (6085): 243--244.

\item [Holloway]G. 1994. Representing eddy forcing in
  models. \textsc{woce} Notes 6 (3): 7--9.

\item [Horikawa]K. 1988. \textit{Nearshore Dynamics and Coastal
  Processes}.  Tokyo: University of Tokyo Press.

\item [Houghton]J.T. 1977. \textit{The Physics of
  Atmospheres}. Cambridge: University Press.

\item [Houghton]J.T., L.G.M. Filho, B.A. Callander, N. Harris,
  A. Kattenberg, and K. Maskell. 1996. \textit{Climate Change 1995:
    The Science of Climate Change}. Cambridge: University Press.

\item [Hoyt]D.V., and K.H. Schatten. 1997. \textit{The Role of the Sun
  in Climate Change}.  Oxford: Oxford University Press.

\item [Huffman]G.J., R.F. Adler, B. Rudolf, U. Schneider, and
  P.R. Keehm. 1995.  Global precipitation estimates based on a
  technique for combining satellite-based estimates, rain gauge
  analysis, and \textsc{nwp} model precipitation
  information. \textit{Journal of Climate} 8: 1284--1295.

\item[Huffman]G.J., R.F. Adler, P.A. Arkin, A. Chang, R. Ferraro,
  A. Gruber, J.  Janowiak, A. McNab, B. Rudolf, and
  U. Schneider. 1997. The Global Precipitation Climatology Project
  (\textsc{gpcp}) combined precipitation data set.  \textit{Bulletin
    of the American Meteorological Society} 78 (1): 5--20.

\item [Ichiye]T., and J. Petersen. 1963. The anomalous rainfall of the
  1957--1958 winter in the equatorial central Pacific arid
  area. \textit{Journal of the Meteorological Society of Japan} Series
  II, 41: 172--182.

\item [International Hydrographic Bureau]1953. Limits of oceans and
  seas, 3rd ed. Special Report No. 53, Monte Carlo.

\item [IPCC] Intergovernmental Panel on Climate
  Change. 2007. \textit{Climate Change 2007: The Physical Science
    Basis. Contribution of Working Group I to the Fourth Assessment
    Report of the Intergovernmental Panel on Climate
    Change}. [Solomon, S., D. Qin, M. Manning, Z. Chen, M. Marquis,
    K.B. Averyt, M. Tignor and H.L. Miller (eds.)]. Cambridge
  University Press, Cambridge, United Kingdom and New York, NY, USA,
  996 pp.

\item [Iselin]C. 1936. A study of the circulation of the Western North
  Atlantic.  \textit{Physical Oceanography and Meteorology}. 6 (4):
  101.

\item [Isemer]H.J., and L. Hasse. 1987. \textit{The Bunker Climate
  Atlas of the North Atlantic}. Volume 2.  Berlin: Springer-Verlag.

\item [Jackett]D.R., and T.J. McDougall. 1997. A neutral density
  variable for the world's oceans. \textit{Journal of Physical
    Oceanography} 27: 237--263.

\item [Jan van Oldenborgh] G., M.A. Balmaseda, L. Ferranti,
  T.N. Stockdale, and D.L.T. Anderson. 2005. Did the ECMWF Seasonal
  Forecast Model Outperform Statistical \textsc{enso} Forecast Models
  over the Last 15 Years? \textit{Journal of Climate} 18 (16):
  3240--3249.

\item [Jarosz]E., D.A. Mitchell, D.W. Wang, and
  W.J. Teague. 2007. Bottom-Up Determination of Air-Sea Momentum
  Exchange Under a Major Tropical Cyclone. \textit{Science} 315
  (5819): 1707--1709.

\item [Jayne]S.R., L.C.S. Laurent and S.T. Gille. 2004. Connections
  between ocean bottom topography and Earth's
  climate. \textit{Oceanography} 17 (1): 65--74.

\item [Jelesnianski]C.P.J., P.C. Chen, and
  W.A. Shaffer. 1992. \textsc{slosh}: Sea, lake, and overland surges
  from hurricanes. \textsc{noaa} Technical Report \textsc{nws} 48.

\item[Jerlov]N.G. 1976. \textit{Marine Optics}. Amsterdam: Elsevier
  Scientific Publishing Company.

\item[Ji]M., A. Leetmaa, and V. Kousky. 1996. Coupled model
  predictions of \textsc{enso} during the 1980s and the 1990s at the
  National Centers for Environmental Prediction. \textit{Journal of
    Climate} 9 (12): 3105--3120.

\item[Ji]M., D.W. Behringer, and A. Leetmaa. 1998. An improved coupled
  model for \textsc{enso} prediction and implications for ocean
  initialization. Part II: The coupled model. \textit{Bulletin of the
    American Meteorological Society} 126 (4): 1022--1034.

\item [Johns]E., D.R. Watts, and H.T. Rossby. 1989. A test of
  geostrophy in the Gulf Stream. \textit{Journal of Geophysical
    Research} 94 (C3): 3211--3222.

\item [Johns]T.C., R.E. Carnell, J.F. Crossley, J.M. Gregory,
  J.F.B. Mitchell, C.A. Senior, S.F.B. Trett, and R.A. Wood. 1997. The
  second Hadley Centre coupled ocean-atmosphere GCM: model
  description, spinup and validation. \textit{Climate Dynamics} 13
  (2): 103--134.

\item[Joseph]J., and H. Sender. 1958. Uber die horizontale diffusion
  im meere.  \textit{Deutsches Hydrographiches Zeitung} 11: 49--77.

\item [Josey]S.A., E.C. Kent, and P.K. Taylor. 1999. New insights into
  the ocean heat budget closure problem from analysis of the
  \textsc{soc} air-sea flux climatology. \textit{Journal of Climate}
  12: 2856--2880.
	
\item [JPOTS]Joint Panel on Oceanographic Tables and
  Standards. 1981. \textit{The practical salinity scale 1978 and the
    international equation of state of seawater 1980}. Paris:
  \textsc{unesco} Technical Papers in Marine Science 36: 25.
	
\item [JPOTS]Joint Panel on Oceanographic Tables and
  Standards. 1991. \textit{Processing of Oceanographic Station
    Data}. Paris: \textsc{unesco}.

\item [Kalnay]E., M. Kanamitsu, R. Kistler, W. Collins, D. Deaven,
  L. Gandin, M. Iredell, S. Saha, G. White, J. Woollen, Y. Zhu,
  M. Chelliah, W. Ebisuzaki, W.  Higgins, J. Janowiak, K.C. Mo,
  C. Ropelewski, J. Wang, A. Leetmaa, R. Reynolds, R. Jenne, and
  D. Joseph. 1996. The \textsc{ncep/ncar} 40--year reanalysis
  project. \textit{Bulletin American Meteorological Society} 77:
  437--471.

\item[Kantha]L.H. 1998. Tides--A modern perspective. \textit{Marine
  Geodesy} 21: 275--297.

\item [Kent]E.C., and P.K. Taylor. 1997. Choice of a Beaufort Scale.
  \textit{Journal of Atmospheric and Oceanic Technology} 14 (2):
  228--242.

\item[Kerr]R.A. 1998. Models win big in forecasting El
  Ni\~{n}o. \textit{Science} 280 (5363): 522--523.

\item [Kerr]R.A. 2002. Salt fingers mix the sea. \textit{Science} 295
  (5561): 1821.

\item [Kiehl]J.T., and K.E. Trenberth. 1996. Earth's annual global
  mean energy budget. \textit{Bulletin of the American Meteorological
    Society} 78 (2): 197--208.

\item [Kilpatrick]K.A., G.P. Podesta, and R. Evans. 2001. Overview of
  the \textsc{noaa/nasa} advanced very high resolution radiometer
  Pathfinder algorithm for sea surface temperature and associated
  matchup database. \textit{Journal of Geophysical Research} 106:
  9179--9198.

\item [Kistler]R.E., E. Kalnay, W. Collins, S. Saha, G. White,
  J. Woolen, M.  Chelliah, and W. Ebisuzaki. 2000. The
  \textsc{ncep/ncar} 50-year reanalysis.  \textit{Bulletin of the
    American Meteorological Society} 82: 247--267.

\item [Komen]G.J., L. Cavaleri, M. Donelan, K. Hasselmann,
  S. Hasselmann, and P.A.E.M. Jans\-sen. 1996. \textit{Dynamics and
    Modelling of Ocean Waves}. 1st paperback ed.  Cambridge:
  University Press.

\item [Kullenberg]B. 1954. Vagn Walfrid Ekman
  1874--1954. \textit{Journal du Conseil international pour
    l'exploration de la mer} 20 (2): 140--143.

\item [Kumar]A., A. Leetmaa, and M. Ji. 1994. Simulations of
  atmospheric variability induced by sea surface temperatures and
  implications for global warming. \textit{Science} 266 (5185):
  632--634.

\item [Kundu]P.K. 1990. \textit{Fluid Mechanics}. San Diego: Academic
  Press.

\item [Kunze]E., and J.M. Toole. 1997. Tidally driven vorticity,
  diurnal shear, and turbulence atop Fieberling
  Seamount. \textit{Journal of Physical Oceanography} 27 (2):
  2,663--2,693.

\item [Lagerloef]G.S.E., G. Mitchum, R. Lukas, and
  P. Niiler. 1999. Tropical Pacific near-surface current estimates
  from altimeter, wind and drifter data.  \textit{Journal of
    Geophysical Research} 104 (C10): 23,313--23,326.

\item [Lamb]H. 1945. \textit{Hydrodynamics}. 6th, first American
  edition. New York: Dover Publications.

\item [Lang]K.R. 1980. \textit{Astrophysical Formulae: A Compendium
  for the Physicist and Astrophysicist}. 2nd ed.  Berlin:
  Springer-Verlag.

\item [Langer]J. 1999. Computing in physics: Are we taking it too
  seriously? Or not seriously enough? \textit{Physics Today} 52 (7):
  11--13.

\item [Langmuir]I. 1938. Surface motion of water induced by
  wind. \textit{Science} 87: 119--123.

\item [Larson]R. 2002. E-Enabled textbooks: Lower cost, higher
  functionality. \textit{Syllabus} 15(10): 44.

\item [Latif]M., A. Sterl, E. Maier-Reimer, and M.M. Junge.
  1993. Structure and predictability of the El Ni\~{n}o/Southern
  Oscillation phenomenon in a coupled ocean-atmosphere
  model. \textit{Journal of Climate} 6: 700--708.

\item [Lawrence]M.G., J. Landgraf, P. Jöckel, and
  B. Eaton. 1999. Artifacts in global atmospheric modeling: Two recent
  examples. \textit{EOS Transactions American Geophysical Union} 80
  (11): 123, 128.

\item [Lean]J., J. Beer, and R. Bradley. 1995. Reconstruction of solar
  irradiance since 1610: Implications for climate
  change. \textit{Geophysical Research Letters} 22 (23): 3195--3198.

\item [Ledwell]J.R., A.J. Watson and C.S. Law 1998. Mixing of a tracer
  in the pycnocline. \textit{Journal of Geophysical Research}
  103(C10): 21,499--421,529.

\item[Leetmaa]A., and A.F. Bunker. 1978. Updated charts of the mean
  annual wind stress, convergences in the Ekman layers and Sverdrup
  transports in the North Atlantic. \textit{Journal of Marine
    Research} 36: 311--322.

\item[Leetmaa]A., J.P. McCreary, and D.W. Moore. 1981. Equatorial
  currents; observation and theory. In: \textit{Evolution of Physical
    Oceanography}. Edited by B. A. Warren and
  C. Wunsch. 184--196. Cambridge: Massachusetts Institute of
  Technology Press.

\item [LeProvost]C., M.L. Genco, F. Lyard, P. Vincent, and
  P. Canceil. 1994. Spectroscopy of the world ocean tides from a
  finite element hydrodynamic model. \textit{Journal Geophysical
    Research} 99 (C12): 24,777--24,797.

\item [LeProvost]C., A. F. Bennett and D. E. Cartwright. 1995. Ocean
  tides for and from Topex/ Poseidon. \textit{Science} 267 (5198):
  639--647.

\item [Levitus]S. 1982. \textit{Climatological Atlas of the World
  Ocean}. \textsc{noaa} Professional Paper 13.

\item [Levitus]S. 1994. \textit{World Ocean Atlas 1994 \textsc{cd-rom}
  Data Set}. \textsc{noaa} National Oceanographic Data Center.

\item [Lewis]E.L. 1980. The Practical Salinity Scale 1978 and its
  antecedents.  \textit{IEEE Journal of Oceanic Engineering}
  \textsc{oe}-5: 3--8.

\item [List]R.J. 1966. \textit{Smithsonian Meteorological Tables}. 6th
  ed. Washington DC: The Smithsonian Institution.

\item [Liu]W.T. 2002. Progress in scatterometer
  application. \textit{Journal of Oceanography} 58: 121--136.

\item [Longuet-Higgins]M.S., and O.M. Phillips. 1962. Phase velocity
  effects in tertiary wave interactions. \textit{Journal of Fluid
    Mechanics.} 12 (3): 333--336.

\item [Lynch]D.R., J.T.C. Ip, C.E. Naimie, and
  F.E. Werner. 1996. Comprehensive coastal circulation model with
  application to the Gulf of Maine.  \textit{Continental Shelf
    Research} 16 (7): 875--906.

\item [Lynn]R.J., and J.L. Reid. 1968. Characteristics and circulation
  of deep and abyssal waters. \textit{Deep-Sea Research} 15 (5):
  577--598.

\item [McAvaney]B.J., C. Covey, S. Joussaume, V. Kattsov, A. Kitoh,
  W. Ogana, A.J. Pitman, A.J. Weaver, R. A. Wood and
  Z.-C. Zhao. 2001. Model Evaluation. In: \textit{Climate Change 2001:
    The Scientific Basis. Contribution of Working Group 1 to the Third
    Assessment Report of the Intergovernmental Panel on Climate
    Change}.  Editeb by J.T. Houghton, Y. Ding, D.J. Griggs,
  N. Noguer, P.J. v. d. Linden, X.  Dai, K. Maskell and
  C.A. Johnson. Cambridge, University Press: 881.

\item [MacKenzie]K.V. 1981. Nine-term equation for sound speed in the
  ocean.  \textit{Journal of the Acoustical Society of America} 70:
  807--812.

\item [McDougall]T.J. 1987. Neutral surfaces. \textit{Journal of
  Physical Oceanography} 17 (11): 1950--1964.

\item [McDougall]T. J. and R. Feistel 2003. What causes the adiabatic
  lapse rate? \textit{Deep Sea Research Part I: Oceanographic Research
    Papers} 50 (12): 1523--1535.

\item [McIntyre]M.E. 1981. On the `wave momentum'
  myth. \textit{Journal of Fluid Mechanics} 106: 331--347.

\item [McNally]G.J., W.C. Patzert, A.D. Kirwan, and
  A.C. Vastano. 1983. The near-surface circulation of the North
  Pacific using satellite tracked drifting buoys. Journal of
  Geophysical Research 88 (C12): 7,507--7,518.

\item [McPhaden]M.J. 1986. The equatorial undercurrent: 100 years of
  discovery. \textit{EOS Transansactions of the Amererican Geophysical
    Union} 67 (40): 762--765.

\item [McPhaden]M.J., A.J. Busalacchi, R. Cheney, J.R. Donguy,
  K.S. Gage, D. Halpern, M. Ji, P. Julian, G. Meyers, G.T. Mitchum,
  P.P. Niiler, J. Picaut, R.W. Reynolds, N. Smith
  K. Takeuchi. 1998. The Tropical Ocean-Global Atmosphere
  (textsc{toa}) observing system: A decade of progress.
  \textit{Journal of Geophysical Research} 103: 14,169--14,240.
	
\item [McPhaden]M.J. 1999. Genesis and evolution of the 1997-1998 El
  Ni\~{n}o. \textit{Science} 283 (5404): 950--954.

\item [McPhaden]M.J., S.E. Zebiak, and
  M.H. Glantz. 2006. \textsc{enso} as an Integrating Concept in Earth
  Science. \textit{Science} 314 (5806): 1740--1745.

\item [Ma]X.C., C.K. Shum, R.J. Eanes, and
  B.D. Tapley. 1994. Determination of ocean tides from the first year
  of Topex/Poseidon altimeter measurements. \textit{Journal of
    Geophysical Research} 99 (C12): 24,809--24,820.

\item [Malanotte-Rizzoli]P., Ed. 1996. \textit{Modern Approaches to
  Data Assimilation in Ocean Modeling}. Amsterdam: Elsevier.

\item [Maltrud]M.E., R.D. Smith, A.J. Semtner, and
  R.C. Malone. 1998. Global eddy-resolving ocean simulations driven by
  1985--1995 atmospheric winds.  \textit{Journal of Geophysical
    Research} 103 (C13): 30,825--30,852.

\item [Margules]M. 1906. Uber Temperaturschichtung in
  stationarbewegter und ruhender Luft. \textit{Meteorologische
    Zeitschrift} 241--244.

\item [Marotzke]J., and J.R. Scott. 1999. Convective mixing and the
  thermohaline circulation. \textit{Journal of Physical Oceanography}
  29 (11): 2962--2970.

\item [Marotzke]J. 2000. Abrupt climate change and thermohaline
  circulation: Mechanisms and predictability. \textit{Proceedings
    National Academy of Sciences} 97 (4): 1347--1350.

\item [Martrat]B., J.O. Grimalt, N.J. Shackleton, L. de Abreu,
  M.A. Hutterli, and T.F. Stocker. 2007. Four Climate Cycles of
  Recurring Deep and Surface Water Destabilizations on the Iberian
  Margin. \textit{Science} 317 (5837): 502--507.

\item[Matth\"{a}us]W. 1969. Zur entdeckungsgeschichte des
  \"{A}quatorialen Unterstroms im Atkantischen Ozean. \textit{Beitrage
    Meereskunde} 23: 37--70.

\item [Maury]M.F. 1855. \textit{Physical Geography of the Sea}.
  Harper.

\item[May]D.A., M.M. Parmenter, D.S. Olszewski, and
  B.D. McKenzie. 1998.  Operational processing of satellite sea
  surface temperature retrievals at the Naval Oceanographic
  Office. \textit{Bulletin of the American Meteorological Society} 79
  (3): 397--407.

\item [Mellor]G.L., and T. Yamada. 1982. Development of a turbulence
  closure model for geophysical fluid problems. \textit{Reviews of
    Geophysics and Space Physics} 20 (4): 851--875.

\item [Mellor]G.L. 1998. \textit{User's Guide for a Three-dimensional,
  Primitive equation, Numerical Ocean Model Version 1998}. Princeton,
  Princeton University: 41.
	
\item [Menard]H.W., and S.M. Smith. 1966. Hypsometry of ocean basin
  provinces.  \textit{Journal of Geophysical Research} 71: 4305--4325.

\item [Mercier]H., M. Arhan and J.R.E. Lutjeharm. 2003. Upper-layer
  circulation in the eastern Equatorial and South Atlantic Ocean in
  January--March 1995. \textit{Deep-Sea Research} 50 (7): 863--887.

\item [Merryfield]W.J., G. Holloway, and A.E. Gargett.  1999. A global
  ocean model with double-diffusion mixing. \textit{Journal of
    Physical Oceanography} 29 (6): 1124--1142.

\item [Miles]J.W. 1957. On the generation of surface waves by shear
  flows.  \textit{Journal of Fluid Mechanics}. 3 (2) 185--204.

\item [Millero]F.J., G. Perron, and J.F. Desnoyers. 1973. Heat
  capacity of seawater solutions from 5\degrees\ to 35\degrees C and
  0.05 to 22$^\circ/_{\circ\circ}$ chlorinity. \textit{Journal of
    Geophysical Research} 78 (21): 4499--4506.

\item[Millero]F.J., C.-T. Chen, A. Bradshaw, and
  K. Schleicher. 1980. A new high pressure equation of state for
  seawater. \textit{Deep-Sea Research} 27A: 255--264.

\item [Millero]F.J., and A. Poisson. 1981. International
  one-atmosphere equation of state of seawater.\textit{Deep-Sea
    Research} 28A (6): 625--629.

\item [Millero]F.J. 1996. \textit{Chemical Oceanography} (2nd ed). New
  York, CRC Press.

\item [Millero]F. J., R. Feistel, et al. 2008. The composition of
  Standard Seawater and the definition of the Reference-Composition
  Salinity Scale. \textit{Deep Sea Research Part I: Oceanographic
    Research Papers} 55 (1): 50--72.

\item [Montgomery]R.B., and E.D. Stroup. 1962. Equatorial Waters and
  Currents at 150\degrees W in July--August, 1952. Baltimore: The
  Johns Hopkins Press.

\item [Morel]A. 1974. Optical porperties of pure water and pure
  seawater. In: \textit{Optical Aspects of Ocean\-ography}. Edited by
  N. G. Jerlov and E. S.  Nielson.  1--24. Academic Press.
	
\item [Moskowitz]L. 1964. Estimates of the power spectrums for fully
  developed seas for wind speeds of 20 to 40 knots. \textit{Journal of
    Geophysical Research} 69 (24): 5161--5179.

\item [Moum]J.N., and D.R. Caldwell. 1985. Local influences on
  shear-flow turbulence in the equatorial ocean. \textit{Science} 230:
  315--316.

\item [Munk]W.H. 1950. On the wind-driven ocean
  circulation. \textit{Journal of Meteorology} 7 (2): 79--93.

\item [Munk]W.H., and E. Palmen. 1951. A note on the dynamics of the
  Antarctic Circumpolar Current. \textit{Tellus} 3: 53--55.

\item [Munk]W.H. 1966. Abyssal recipes. \textit{Deep-Sea Research} 13:
  707--730.

\item [Munk]W.H., G.R. Miller, F.E. Snodgrass, and N.F. Barber. 1963.
  Directional recording of swell from distant
  storms. \textit{Philosophical Transactions Royal Society of London}
  255 (1062): 505--584.

\item [Munk]W.H., and D.E. Cartwright. 1966. Tidal spectroscopy and
  prediction.  \textit{Philosophical Transactions Royal Society
    London} Series A. 259 (1105): 533--581.

\item [Munk]W.H., R.C. Spindel, A. Baggeroer, and
  T.G. Birdsall. 1994. The Heard Island feasibility
  test. \textit{Journal of the Acoustical Society of America} 96 (4):
  2330--2342.

\item [Munk]W., P. Worcester, and C. Wunsch. 1995. \textit{Ocean
  Acoustic Tomography}.  Cambridge: University Press.

\item [Munk]W. and C. Wunsch 1998. Abyssal recipes
  II. \textit{Deep-Sea Research} 45: 1976--2009.

\item [NAS]National Academy of Sciences. 1963. \textit{Ocean Wave
  Spectra: Proceedings of a Conference}.  Englewood Cliffs, New
  Jersey: Prentice-Hall.

\item [Neal]V.T., S. Neshyba, and W. Denner. 1969. Thermal
  stratification in the Arctic Ocean. \textit{Science} 166 (3903):
  373--374.

\item[Nerem]R.S., E. Leuliette, and A. Cazenave. 2006. Present-day
  sea-level change: A review. \textit{Comptes Rendus Geosciences} 338
  (14--15): 1077--1083.
	
\item [Neumann]G., and W.J. Pierson.  1966. \textit{Principles of
  Physical Oceanography}.  New Jersey: Prentice-Hall.

\item [Newton]P. 1999. A manual for planetary
  management. \textit{Nature} 400 (6743): 399.

\item [Niiler]P.P., R.E. Davis, and H.J. White. 1987. Water-following
  characteristics of a mix\-ed layer drifter. \textit{Deep-Sea
    Research} 34 (11): 1867--1881.

\item [Niiler]P.P., A.S. Sybrandy, K. Bi, P.M. Poulain, and
  D. Bitterman. 1995.  Measurement of the water following capability
  of holey-sock and \textsc{tristar} drifters. \textit{Deep-Sea
    Research} 42 (11/12): 1951--1964.

\item [North]G.R. and S. Nakamoto. 1989. Formalism for comparing rain
  estimation designs. \textit{Journal of Atmospheric and Oceanic
    Technology}. 6 (6): 985--992.

\item [Oberhuber]J.M. 1988. An atlas based on the \textsc{coads} data
  set: The budgets of heat, buoyancy and turbulent kinetic energy at
  the surface of the global ocean. \textit{Max-Planck-Institut f\"{u}r
    Meteorologie}: Report 15.

\item [Open University]1989a. \textit{Ocean Circulation}.  Oxford:
  Pergamon Press.

\item[Open University]1989b. \textit{Seawater: Its Composition,
  Properties and Behaviour}.  Oxford: Pergamon Press.

\item [Open University]1989c. \textit{Waves, Tides and Shallow
  Water--Processes}.  Oxford: Pergamon Press.

\item [Oppenheim]A.V. and R.W. Schafer. 1975. \textit{Digital Signal
  Processing}.  Englewood Cliffs, New Jersey: Prentice-Hall.

\item [Orsi]A.H., T. Whitworth and W.D. Nowlin. 1995. On the
  meridional extent and fronts of the Antarctic Circumpolar
  Current. \textit{Deep-Sea Research} 42(5): 641--673.

\item [Orsi]A.H., G.C. Johnson, and J.L. Bullister. 1999. Circulation,
  mixing, and production of Antarctic Bottom Water.  \textit{Progress
    in Oceanography} 43 55-109.

\item [Pacanowski]R., and S.G.H. Philander. 1981. Parameterization of
  vertical mixing in numerical models of tropical
  oceans. \textit{Journal of Physical Oceanography} 11: 1443--1451.

\item [Pacanowski]R.C., and S.M. Griffies. 1999 \textit{MOM 3.0
  Manual}.  \textsc{Noaa}/Geophysical Fluid Dynamics Laboratory,
  Princeton.

\item [Parke]M.E., R.H. Stewart, D.L. Farless, and
  D.E. Cartwright. 1987. On the choice of orbits for an altimetric
  satellite to study ocean circulation and tides. \textit{Journal of
    Geophysical Research} 92: 11,693--11,707.

\item [Parker]R.L. 1994. \textit{Geophysical Inverse
  Theory}. Princeton: Princeton University Press.

\item [Pedlosky]J. 1987. \textit{Geophysical Fluid Dynamics}. 2nd ed.
  Berlin: Springer Verlag.

\item [Pedlosky]J. 1996. \textit{Ocean Circulation Theory}.  Berlin:
  Springer--Verlag.

\item [Percival]D.B., and A.T. Walden. 1993. \textit{Spectral Analysis
  for Physical Applications: Multitaper and Conventional Univariate
  Techniques}.  Cambridge: University Press.

\item [Philander]S.G.H., T. Yamagata, and
  R.C. Pacanowski. 1984. Unstable air-sea interactions in the
  tropics. \textit{Journal of Atmospheric Research} 41: 604--613.

\item [Philander]S.G. 1990. \textit{El Ni\~{n}o, La Ni\~{n}a, and the
  Southern Oscillation}. Academic Press.

\item [Phillips]O.M. 1957. On the generation of waves by turbulent
  wind.  \textit{Journal of Fluid Mechanics}. 2 (5): 417--445.

\item [Phillips]O.M. 1960. On the dynamics of unsteady gravity waves
  of finite amplitude.  Part I.  The elementary
  interactions. \textit{Journal of Fluid Mechanics} 9 (2): 193--217.

\item [Picaut]J., F. Masia, and Y.d. Penhoat. 1997. An
  advective-reflective conceptual model for the oscillatory nature of
  the ENSO. \textit{Science} 277 (5326): 663--666.

\item [Pickard]G.L., and W.J. Emery. 1990. \textit{Descriptive
  Physical Oceanography: An Introduction}. 5th enlarged ed. Oxford:
  Pergamon Press.

\item [Pierson]W.J., and L. Moskowitz. 1964. A proposed spectral form
  for fully developed wind seas based on the similarity theory of
  S.A. Kitaigordskii.  \textit{Journal of Geophysical. Research} 69:
  5181--5190.

\item [Pinet]P.R. 2000. \textit{Invitation to oceanography}. 2nd
  Edition.  Sudbury, Massachusetts: Jones and Bartlett Publishers.

\item [Polzin]K.L., J.M. Toole, J.R. Ledwell, and
  R.W. Schmitt. 1997. Spatial variability of turbulent mixing in the
  abyssal ocean. \textit{Science} 276 (5309): 93--96.

\item[Post]D.E. and L.G. Votta. 2005. Computational science demands a
  new paradigm. \textit{Physics Today} 58 (1): 35--41.
	
\item [Powell]M. D., P.J. Vickery and T.A. Reinhold. 2003. Reduced
  drag coefficient for high wind speeds in tropical
  cyclones. \textit{Nature} 422 (6929): 279--283.

\item [Press]W.H., S.A. Teukolsky, W.T. Vetterling, and
  B.P. Flannery. 1992.  \textit{Numerical Recipes in FORTRAN}. 2nd ed.
  Cambridge: University Press.

\item [Preston-Thomas]H. 1990. The International Temperature Scale of
  1990 (\textsc{its}-90). \textit{Metrologia} 27 (1): 3--10.

\item [Proudman]J. 1916. On the motion of solids in a liquid
  possessing vorticity. \textit{Proceedings Royal Society (London)} A
  92: 408--424

\item [Pugh]D.T. 1987. \textit{Tides, Surges, and Mean Sea-Level}.
  Chichester: John Wiley.

\item [Rahmstorf]S. 1995. Bifurcations of the Atlantic thermohaline
  circulation in response to changes in the hydrological
  cycle. \textit{Nature} 378: 145--149.

\item [Ralph]E.A., and P.P. Niiler. 2000. Wind-driven currents in the
  tropical Pacific. \textit{Journal of Physical Oceanography} 29 (9):
  2121--2129.

\item [Ramanathan]V., B. Subasilar, G.J. Zhang, W. Conant, R.D. Cess,
  J.T. Kiehl, H. Grassl, and L. Shi. 1995. Warm pool heat budget
  shortwave cloud forcing: A missing physics? \textit{Science} 267
  (5197): 499--503.

\item [Rasmusson]E.M., and T.H. Carpenter. 1982. Variations in
  tropical sea surface temperature and surface wind fields associated
  with the Southern Oscillation/El Ni\~{n}o. \textit{Monthly Weather
    Review} 110: 354--384.

\item [Rasmusson]E.M., and J.M. Wallace. 1983. Meteorological Aspects
  of the El Ni\~{n}o/Southern Oscillation. \textit{Science} 222:
  1195--1202.

\item [Reid]R.O. 1948. The equatorial currents of the eastern Pacific
  as maintained by the stress of the wind. \textit{Journal of Marine
    Research} 7 (2): 75--99.

\item [Reynolds]O. 1883. An experimental investigation of the
  circumstances which determine whether the motion of water will be
  direct or sinuous, and the law of resistance in parallel
  channels. \textit{Philosophical Transactions, Royal Society London}
  174: 935.

\item [Reynolds]R.W., and D.C. Marsico. 1993. An improved real-time
  global sea surface temperature analysis. \textit{Journal of Climate}
  6: 114--119.

\item [Reynolds]R.W., and T.M. Smith. 1994. Improved global sea
  surface temperature analysis using optimum
  interpolation. \textit{Journal of Climate} 7: 929--948.

\item [Reynolds]R.W., and T.M. Smith. 1995. A high-resolution global
  sea surface climatology. \textit{Journal of Climate} 8 (6):
  1571--1583.

\item [Reynolds]R.W., N.A. Rayner, T. M. Smith, D. C. Stokes,
  W. Wang. 2002. An improved in situ and satellite SST analysis for
  climate. \textit{Journal of Climate} 15: 1609--1625.

\item [Rhines]P.B. 1984. Notes on the general circulation of the
  ocean. In: \textit{A Celebration in Geophysics and
    Oceanography--1982}. 83--86. Scripps Institution of Oceanography.

\item [Richardson]E.G. 1961. \textit{Dynamics of Real Fluids}. 2nd
  ed. London: Edward Arnolds.

\item [Richardson]P.L. 1981. Eddy kinetic energy in the North Atlantic
  from surface drifters. \textit{Journal of Geophysical Research}. 88
  (C7): 4355--4367.

\item [Richardson]P.L. 1993. Tracking Ocean Eddies. \textit{American
  Scientist} 81: 261--271.

\item [Richardson]P. L. 2008. On the history of meridional overturning
  circulation schematic diagrams. \textit{Progress In Oceanography} 76
  (4): 466--486.

\item [Ring Group]R.H. Backus, G.R. Flierl, D.R. Kester, D.B. Olson,
  P.L. Richardson, A.C. Vastano, P.H. Wiebe, and
  J.H. Wormuth. 1981. Gulf Stream cold-core rings: Their physics,
  chemistry, and biology. \textit{Science} 212 (4499): 1091--1100.

\item [Rintoul]S.R., and C. Wunsch. 1991. Mass, heat, oxygen and
  nutrient fluxes and budgets in the North Atlantic
  Ocean. \textit{Deep-Sea Research} 38 (Supplement 1): S355--S377.

\item [Riser]S. C., L. Ren, et al. 2008. Salinity in
  ARGO. \textit{Oceanography} 21 (1): 56--67.

\item [Robinson]A.R., S.M. Glenn, M.A. Spall, L.J. Walstad,
  G.M. Gardner, and W.G. Leslie. 1989. Forecasting Gulf Stream
  meanders and rings. \textit{EOS Transactions American Geophysical
    Union} 70: (45).

\item [Roed]L.P., B. Hackett, B. Gjevik, and L.I. Eide. 1995. A review
  of the Metocean Modeling Project (\textsc{mompop}) Part 1: Model
  comparison study. In: \textit{Quantitative Skill Assessment for
    Coastal Ocean Models}. Edited by D. R. Lynch and
  A. M. Davies. 285--305. Washington DC: American Geophysical Union.

\item [Ropelewski]C.F., and M.S. Halpert. 1987. Global and regional
  precipitation associated with El Ni\~{n}o/Southern
  Oscillation. \textit{Monthly Weather Review} 115: 1606--1626.

\item [Rossby]C.C. 1936. Dynamics of steady ocean currents in the
  light of experimental fluid mechanics. \textit{Papers in Physical
    Oceanography and Meteorology, Massachusetts Institute of
    Technology and Woods Hole Oceanographic Institution}. 5(1): 43.

\item [Rossow]W.B. and R.A. Schiffer. 1991. \textsc{Isccp} Cloud Data
  Products. \textit{Bulletin of the American Meteorology Society}, 72
  (1): 2--20.

\item [Rudnick], D.L., T.J. Boyd, R.E. Brainard, G.S. Carter,
  G.D. Egbert, M.C. Gregg, P.E. Holloway, J.M. Klymak, E. Kunze,
  C.M. Lee, M.D. Levine, D.S. Luther, J.P. Martin, M.A. Merrifield,
  J.N. Moum, J.D. Nash, R. Pinkel, L. Rainville,
  T.B. Sanford. 2003. From Tides to Mixing Along the Hawaiian
  Ridge. \textit{Science} 301 (5631): 355--357.

\item [Sabine]C.L., R.A. Feely, N. Gruber, R.M. Key, K. Lee,
  J.L. Bullister, R. Wanninkhof, C.S. Wong, D.W.R. Wallace,
  B. Tilbrook, F.J. Millero, T.-H. Peng, A. Kozyr, T. Ono and
  A.F. Rios. 2004. The Oceanic Sink for Anthropogenic
  CO$_2$. \textit{Science} 305 (5682): 367--371.

\item [Sandwell]D.T., and W.H.F. Smith. 2001. Bathymetric
  Estimation. In: \textit{Satellite Altimetry and Earth
    Sciences}. Edited by L.-L. Fu and A. Cazanave. 441--457. San
  Diego: Academic Press.

\item [Satake]K., K. Shimazaki, Y. Tsuji, and K. Ueda. 1996. Time and
  size of a giant earthquake in Cascadia inferred from Japanese
  tsunami records of January 1700. \textit{Nature} 379 (6562):
  246--249.

\item [Saunders]P.M. and N. P. Fofonoff. 1976. Conversion of pressure
  to depth in the ocean. \textit{Deep-Sea Research} 23: 109--111.

\item [Saunders]P.M. 1986. The accuracy of measurements of salinity,
  oxygen and temperature in the deep ocean. \textit{Journal of
    Physical Oceanography} 16 (1): 189--195.

\item [Schmitt]R.W., H. Perkins, J.D. Boyd, and
  M.C. Stalcup. 1987. C-SALT: An investigation of the thermohaline
  staircase in the western tropical North Atlantic. \textit{Deep-Sea
    Research} 34 (10): 1655--1665.

\item [Schmitt]R.W., P.S. Bogden, and C.E. Dorman. 1989. Evaporation
  minus precipitation and density fluxes for the North
  Atlantic. \textit{Journal of Physical Oceanography} 19: 1208--1221.

\item [Schmitt]R.W. 1994. The ocean freshwater cycle. \textsc{jsc}
  Ocean Observing System Development Panel, Texas A\&M University,
  College Station, Texas: 40.

\item [Schmitz]W.J. 1996. On the World Ocean Circulation: Volume
  I. Some Global Features/ North Atlantic Circulation. Woods Hole
  Oceanographic Institution, Technical Report \textsc{whoi}--96--03.

\item [Schubert]S.D., R.B. Rood, and J. Pfaendtner. 1993. An
  assimilated dataset for Earth science applications. \textit{Bulletin
    American Meteorological Society} 74 (12): 2331--2342.

\item [Selby]J.E.A., and R.A. McClatchey. 1975. Atmospheric
  transmittance from 0.25 to 28.5 $\mu$m: Computer code
  \textsc{lowtran} 3. Air Force Cambridge Research Laboratories,
  Optical Physics Laboratory Technical Report TR--75--0255.

\item [Service]R.F. 1996. Rock chemistry traces ancient
  traders. \textit{Science} 274 (5295): 2012--2013.

\item[Sette]O.E., and J.D. Isaacs. 1960. Symposium on ``The Changing
  Pacific Ocean in 1957 and 1958''. California Cooperative Oceanic
  Fisheries Investigations Reports VII: 13--217.

\item [Shamos]M.H. 1995. \textit{The Myth of Scientific Literacy}. New
  Brunswick: Rutger University Press.

\item [Shchepetkin]A. F. and J.C. McWilliams. 2004. The Regional
  Oceanic Modeling System: A split-explicit, free-surface,
  topography-following-coordinate ocean model. \textit{Ocean
    Modelling} 9: 347--404.

\item [Shepard]F.P., G.A. MacDonald and D.C. Cox. 1950. The tsunami of
  April 1, 1946. \textit{Bulletin of the Scripps Institution of
    Oceanography} 5(6): 391--528.

\item [Shepard]F.P. 1963. \textit{Submarine Geology}. 2nd ed. New
  York: Harper and Row.

\item [Slingo]J.M., K.R. Sperber, J.S. Boyle, J.-P. Ceron, M. Dix,
  B. Dugas, W. Ebisuzaki, J. Fyfe, D. Gregory, J.-F. Gueremy, J. Hack,
  A. Harzallah, P. Inness, A. Kitoh, W.K.-M. Lau, B. McAvaney,
  R. Madden, A. Matthews, T.N. Palmer, C.-K. Park, D. Randall, and
  N. Renno. 1995. Atmospheric Model Intercomparison Project
  (\textsc{amip}): Intraseasonal Oscillations in 15 Atmospheric
  General Circulation Models (Results From an \textsc{aimp}
  Diagnostics Subproject). World Meteorological Organization/World
  Climate Research Programme, WCRP--88 (WMO/TD No. 661).

\item [Slutz]R.J., S.J. Lubker, J.D. Hiscox, S.D. Woodruff,
  R.L. Jenne, D.H. Joseph, P.M. Steurer, and J.D. Elms. 1985:
  Comprehensive Ocean-Atmosphere Data Set; Release 1. \textsc{noaa}
  Environmental Research Laboratories, Climate Research Program,
  Boulder, Colorado:, 268. (NTIS PB86--105723).

\item [Smagorinski]J. 1963. General circulation experiments with
  primitive equations I. The basic experiment. \textit{Monthly Weather
    Review} 91: 99--164.

\item [Smith]R.D., M.E. Maltrud, F.O. Bryan and
  M.W. Hecht. 2000. Numerical simulation of the North Atlantic ocean
  at 1/10\degrees. \textit{Journal of Physical Oceanography} 30 (7):
  1532--1561.

\item [Smith]S.D. 1980. Wind stress and heat flux over the ocean in
  gale force winds. \textit{Journal of Physical Oceanography} 10:
  709-726.
	
\item [Smith]S.D. 1988. Coefficients for sea surface wind stress, heat
  flux and wind profiles as a function of wind speed and
  temperature. \textit{Journal of Geophysical Research} 93:
  15,467--15,472.

\item [Smith]T.M., T.R. Karl and R.W. Reynolds. 2002. How accurate are
  climate simulations? \textit{Science} 296 (5567): 483--484.

\item [Smith]T.M., and R.W. Reynolds. 2004. Improved extended
  reconstruction of \textsc{sst} (1854--1997). \textit{Journal of
    Climate} 17 (6): 2466--2477.
	
\item [Smith]W.H.F., and D.T. Sandwell. 1994. Bathymetric prediction
  from dense satellite altimetry and sparse shipboard
  bathymetry. \textit{Journal of Geophysical Research} 99 (B11):
  21,803--21,824.

\item [Smith]W.H.F. and D.T. Sandwell 1997. Global sea floor
  topography from satellite altimetry and ship depth
  soundings. \textit{Science} 277 (5334): 1956--1962.

\item[Snodgrass]F.E. 1964. Precision digital tide
  gauge. \textit{Science} 146 (3641): 198--208.

\item [Soulen]R.J., and W.E. Fogle. 1997. Temperature scales below 1
  kelvin. \textit{Physics Today} 50 (8 Part 1): 36--42.

\item [Stammer]D., R. Tokmakian, A. Semtner, and C. Wunsch. 1996. How
  well does a 1/4\degrees\ global circulation model simulate
  large-scale oceanic observations? \textit{Journal of Geophysical
    Research} 101 (C10): 25,779--25,811.

\item [Starr]V.P. 1968. \textit{Physics of Negative Viscosity
  Phenomena}.  New York: McGraw-Hill.

\item [Steig]E.J. 2006. Climate change: The south-north
  connection. \textit{Nature} 444 (7116): 152--153.

\item [Stern]M.E. 1960. The `salt fountain' and thermohaline
  convection. \textit{Tellus} 12: 172--175.

\item [Stewart]I. 1992. Warning---handle with care! \textit{Nature}
  \textbf{355}: 16--17.

\item [Stewart]R.H. 1980. Ocean wave measurement techniques. In:
  \textit{Air Sea Interaction, Instruments and Methods}. Edited by
  L. H. F. Dobson and R. Davis.  447--470. New York: Plenum Press.

\item [Stewart]R.H. 1985. \textit{Methods of Satellite
  Oceanography}. University of California Press.

\item [Stewart]R.H. 1995. Predictability of Texas Rainfall Patterns on
  Time Scales of Six to Twelve Months: A Review. In: \textit{The
    Changing Climate of Texas: Predictability and Implications for the
    Future} Edited by J. Norwine, J.R.  Giardino, G.R. North and
  J.B. Valdes. College Station, Texas: Texas A\&M University. 38--47

\item [Stocker]T.F., and O. Marchal. 2000. Abrupt climate change in
  the computer: Is it real? \textit{Proceedings National Academy of
    Sciences} 97 (4): 1362--1365.

\item [Stokes]G.G. 1847. On the theory of oscillatory
  waves. \textit{Cambridge Transactions} 8: 441--473.

\item [Stommel]H. 1948. The westward intensification of wind-driven
  ocean currents. \textit{Transactions, American Geophysical Union} 29
  (2): 202--206.

\item[Stommel]H. 1958. The abyssal circulation. \textit{Deep-Sea
  Research} 5 (1): 80--82.

\item [Stommel]H., A.B. Arons, and A.J. Faller. 1958. Some examples of
  stationary flow patterns in bounded basins. \textit{Tellus} 10 (2):
  179--187.

\item [Stommel]H., and A.B. Arons. 1960. On the abyssal circulation of
  the world ocean---II. An idealized model of the circulation pattern
  and amplitude in oceanic basins. \textit{Deep-Sea Research} 6:
  217--233.

\item [Stommel]H.M., and D.W. Moore. 1989. \textit{An Introduction to
  the Coriolis Force}.  Cambridge: University Press.

\item [Stott]P.A., S.F.B. Tett, G.S. Jones, M.R. Allen,
  J.F.B. Mitchell and G.J. Jenkins 2000. External Control of 20th
  Century Temperature by Natural and Anthropogenic
  Forcings. \textit{Science} 290 (5499): 2133-2137.

\item [Strub]P.T., T.K. Chereskin, P.P. Niiler, C. James, and
  M.D. Levine.  1997. Altimeter-derived variability of surface
  velocities in the California Current System 1. Evaluation of
  \textsc{Topex} altimeter velocity resolution.  \textit{Journal of
    Geophysical Research} 102 (C6): 12,727--12,748.

\item [SUN]Working Group on Symbols, Units and Nomenclature in
  Physical Oceanography. 1985. \textit{The International System of
    units (SI) in oceanography}. \textsc{iapso} Paris: \textsc{Unesco}
  Technical Papers in Marine Science 45: 124.

\item [Sverdrup]H.U. 1947. Wind-driven currents in a baroclinic ocean:
  with application to the equatorial currents of the eastern Pacific.
  \textit{Proceedings of the National Academy of Sciences} 33 (11):
  318--326.

\item [Sverdrup]H.U., M.W. Johnson, and
  R.H. Fleming. 1942. \textit{The Oceans: Their Physics, Chemistry,
    and General Biology}.  Englewood Cliffs, New Jersey:
  Prentice-Hall.

\item [SWAMP Group]Sea Wave Modeling Project. 1985. \textit{Ocean Wave
  Modeling}. New York: Plenum Press.

\item [Swenson]K.R., and A.E. Shaw. 1990. The Argos system: Monitoring
  the world's environment. \textit{Oceanography} 3 (1): 60--61.

\item [Takayabu]Y.N., T. Ihuchi, M. Kachi, A. Shibata, and
  H. Kanzawa. 1999. Abrupt termination of the 1997--98 El Nino in
  response to a Madden-Julian oscillation. \textit{Nature} 402 (6759):
  279--282.

\item [Tapley]B.D., and M.-C. Kim. 2001. Applications to geodesy. In
  \textit{Satellite Altimetry and Earth Sciences}. Edited by L.-L. Fu
  and A.  Cazenave. 371--406. San Diego: Academic Press.

\item [Taylor]G.I. 1921. Experiments with rotating
  fluids. \textit{Proceedings Royal Society (London)} A 100: 114--121.

\item [Taylor]P.K. Editor. 2000. Intercomparison and validation of
  ocean-atmosphere energy flux fields: Final Report of the Joint World
  Climate Research Program and Scientific Committee on Ocean Research
  Working Group on Air-sea Fluxes, \textit{World Climate Research
    Program Report} \textsc{wcrp}-112: 303.
	
\item [Taylor]P. K., E.F. Bradley, C.W. Fairall, D. Legler,
  J. Schultz, R.A. Weller and G.H. White. 2001. Surface fluxes and
  surface reference sites. In: \textit{Observing the Oceans in the
    21st Century}. Edited by C.J. Koblinsky and N.R. Smith. Melbourne,
  Bureau of Meteorology: 177--197.

\item [Tchernia]P. 1980. \textit{Descriptive Regional
  Oceanography}. Oxford: Pergamon Press.

\item [Tennekes]H., and J.L. Lumley. 1990. \textit{A First Course in
  Turbulence}.  Boston: MIT Press.

\item [Thurman]H.V. 1985. \textit{Introductory Oceanography}. Fourth
  ed. Columbus: Charles E. Merrill Publishing Company.

\item [Titov]V.V. and F.I. Gonzalez. 1997. Implementation and testing
  of the Method of Splitting Tsunami (\textsc{most})
  model. \textsc{Noaa} Pacific Marine Environmental Laboratory
  Contribution 1927: 14.

\item [Toggweiler]J.R. 1994. The ocean's overturning
  circulation. \textit{Physics Today} 47 (11): 45--50.

\item [Toggweiler]J. R. and J. Russell 2008. Ocean circulation in a
  warming climate. \textit{Nature} 451 (7176): 286--288.
	
\item [Tolmazin]D. 1985. \textit{Elements of Dynamic
  Oceanography}. Boston: Allen and Unwin.

\item [Tomczak]M., and J.S. Godfrey. 1994. \textit{Regional
  Oceanography: An Introduction}. London: Pergamon.

\item [Tomczak]M. 1999. Some historical, theoretical and applied
  aspects of quantitative water mass analysis. \textit{Journal of
    Marine Research} 57 (2): 275--303.

\item [Trenberth]K.E., and D.J. Shea. 1987. On the evolution of the
  Southern Oscillation. \textit{Monthly Weather Review} 115:
  3078--3096.

\item [Trenberth]K.E., W.G. Large, and J.G. Olson. 1989. The effective
  drag coefficient for evaluating wind stress over the
  oceans. \textit{Journal of Climate}, 2: 1507--1516.

\item [Trenberth]K.E., W.G. Large, and J.G. Olson. 1990. The mean
  annual cycle in global ocean wind stress. \textit{Journal of
    Physical Oceanography} 20 (11): 1742--1760.

\item [Trenberth]K.E., and A. Solomon. 1994. The global heat balance:
  heat transports in the atmosphere and ocean. \textit{Climate
    Dynamics} 10 (3): 107--134.

\item [Trenberth]K.E. 1997. The use and abuse of climate
  models. \textit{Nature} 386 (6621): 131--133.

\item [Trenberth]K.E. 1997. The definition of El
  Ni\~{n}no. \textit{Bulletin of the American Meteorological Society}
  78 (12): 2771--2777.

\item [Trenberth]K.E. and J.M. Caron. 2001. Estimates of meridional
  atmospheric and oceanic heat transports. \textit{Journal of Climate}
  14 (16): 3433--3443.

\item [Uchida]H., S. Imawaki, and J.-H. Hu. 1998. Comparisons of
  Kuroshio surface velocities derived from satellite altimeter and
  drifting buoy data. \textit{Journal of Oceanography} 54: 115--122.

\item [Uppala]S.M., and P.W. K\r{a}llberg, A.J. Simmons, U. Andrae,
  V. Da Costa Bechtold, M. Fiorino, J. K. Gibson, J. Haseler,
  A. Hernandez, G. A. Kelly, X. Li, K. Onogi, S. Saarinen, N. Sokka,
  R. P. Allan, E. Andersson, K. Arpe, M. A. Balmaseda,
  A. C. M. Beljaars, L. Van De Berg, J. Bidlot, N. Bormann, S. Caires,
  F. Chevallier, A. Dethof, M. Dragosavac, M. Fisher, M. Fuentes,
  S. Hagemann, E. H\'{o}lm, B. J. Hoskins, L. Isaksen,
  P.A.E.M. Janssen, R. Jenne, A. P. McNally, J.-F. Mahfouf,
  J.-J. Morcrette, N. A. Rayner, R. W. Saunders, P. Simon, A. Sterl,
  K. E. Trenberth, A. Untch, D. Vasiljevic, P. Viterbo,
  J. Woollen. 2005. The \textsc{era}--40
  re-analysis. \textit{Quarterly Journal of the Royal Meteorological
    Society} 131(612): 2961--3012.

\item [Ursell]F. 1950. On the theoretical form of ocean swell on a
  rotating earth. \textit{Monthly Notices Royal Astronomical Society,
    Geophysical Supplement} 6 (1): G1--G8.

\item [van Meurs]P. 1998. Interactions between near-inertial mixed
  layer currents and the meso\-scale: The importance of spatial
  variability in the vorticity field. \textit{Journal of Physical
    Oceanography} 28 (7): 1363--1388.

\item [Vesecky]J.F., and R.H. Stewart. 1982. The observation of ocean
  surface phenomena using imagery from the Seasat synthetic aperture
  radar: An assessment.  \textit{Journal of Geophysical Research} 87
  (C5): 3397--3430.

\item [von Arx]W.S. 1962. \textit{An Introduction to Physical
  Oceanography}. Reading, Massachusetts: Addison-Wesley.

\item [WAMDI Group]S. Hasselmann, K. Hasselmann, E. Bauer,
  P.A.E.M. Jans\-sen, G.J. Komen, L. Bertotti, P. Lionello,
  A. Guillaume, V.C. Cardone, J.A. Greenwood, M. Reistad,
  L. Zambresky, and J.A. Ewing. 1988. The WAM model---A third
  generation wave prediction model. \textit{Journal of Physical
    Oceanography} 18: 1775--1810.

\item [Warren]B.A. 1973. Transpacific hydrographic sections at
  Latitudes 43\degrees\ S and 28\degrees\ S: The \textsc{scorpio}
  Expedition--II. Deep Water. \textit{Deep-Sea Research} 20: 9--38.
	
\item [WCRP]World Climate Research Program. 1995. \textit{Proceedings
  of the Workshop on Global Coupled General Circulation Models}. World
  Meteorological Organization/World Climate Research Program, WCRP-87
  (WMO/TD Number 655).

\item [Weaver]A.J. and C. Hillaire-Marcel 2004. Global Warming and the
  Next Ice Age. \textit{Science} 304 (5669): 400--402.

\item [Webb]D.J., and N. Suginohara. 2001. Vertical mixing in the
  ocean.  \textit{Nature} 409 (6816): 37.

\item [Webster]F. 1968. Observations of inertial-period motions in the
  deep sea. Reviews of Geophysics 6 (4): 473--490.

\item [Webster]P.J., and R. Lukas. 1992. \textsc{toga coare}: The
  Coupled Ocean-Atmosphere Response Experiment. \textit{Bulletin
    American Meteorological Society} 73 (9): 1377--1416.

\item [Weller]R.A., J.P. Dean, J. Marra, J.F. Price, E.A. Francis, and
  D.C.  Boardman. 1985. Three--dimensional flow in the upper
  ocean. \textit{Science} 227: 1552--1556.

\item [Weller]R.A., and A.J. Plueddmann. 1996. Observations of the
  vertical structure of the oceanic boundary layer. \textit{Journal of
    Geophysical Research} 101 (C4): 8,789--8,806.

\item [Wentz]P.J., S. Peteherych, and L.A. Thomas. 1984. A model
  function for ocean radar cross sections at 14.6 GHz. \textit{Journal
    of Geophysical Research} 89 (C3): 3689--3704.

\item [West]G.B. 1982. Mean Earth ellipsoid determined from Seasat 1
  altimetric observations. \textit{Journal of Geophysical Research} 87
  (B7): 5538--5540.

\item [White]G. Editor. 1996. \textsc{wcrp} Workshop on Air-Sea Flux
  Fields for Forcing Ocean Models and Validating GCMs. Geneva: World
  Meteorological Organization Report WCRP--95 (WMO/TD Number 762).

\item [White]W.B., and D.R. Cayan. 1998. Quasi-periodicity and global
  symmetries in interdecadal upper ocean temperature
  variability. \textit{Journal of Geophysical Research} 103 (C10):
  21,335--21,354.

\item [Whitham]G.B. 1974. \textit{Linear and Nonlinear Waves}.  New
  York: John Wiley.

\item [Whittaker]E.T., and G.N. Watson. 1963. \textit{A Course of
  Modern Analysis}. 4th ed.  Cambridge: University Press.

\item [Whitworth]T., and R.G. Peterson. 1985. Volume transport of the
  Antarctic Circumpolar Current from bottom pressure
  measurements. \textit{Journal of Physical Oceanography} 15 (6):
  810--816.

\item [Wiegel]R.L. 1964. \textit{Oceanographical Engineering}.
  Englewood Cliffs, New Jersey: Prentice Hall.

\item [Wilson]W.D. 1960. Equation for the speed of sound in sea
  water. \textit{Journal of the Acoustical Society of America} 32
  (10): 1357.

\item [Woodruff]S.D., R.J. Slutz, R.L. Jenne, and
  P.M. Steurer. 1987. A comprehensive ocean--atmosphere data
  set. \textit{Bulletin American Meteorological Society} 68:
  1239--1250.

\item [Wooster]W.S. 1960. Investigations of equatorial
  undercurrents. \textit{Deep-Sea Research} 6 (4): 263--264.

\item [Wooster]W.S., A.J. Lee, and G. Dietrich. 1969. Redefinition of
  salinity.  \textit{Deep-Sea Research} 16 (3): 321--322.

\item [Worthington]L.V. 1981. The water masses of the World Ocean:
  Some results of a fine-scale census. In: \textit{Evolution of
    Physical Oceanography: Scientific surveys in honor of Henry
    Stommel}. Edited by B. A. Warren and C. Wunsch. 42--69. Cambridge:
  Massachusetts Institute of Technology.

\item [Wunsch]C. 1996. \textit{The Ocean Circulation Inverse Problem}.
  Cambridge: University Press.

\item [Wunsch]C. 2002. Ocean observations and the climate forecast
  problem.In: \textit{Meteorology at the Millennium}, Edited by
  R.P. Pearce. London: Royal Meteorological Society: 233--245.

\item [Wunsch]C. 2002b. What is the thermohaline circulation?
  \textit{Science} 298 (5596): 1179--1180.

\item [Wust]G. 1964. The major deep--sea expeditions and research
  vessels 1873--1960. \textit{Progress in Oceanography} 2: 3--52.

\item [Wyrtki]K. 1975. El Ni\~{n}o---The dynamic response of the
  equatorial Pacific Ocean to atmospheric forcing. \textit{Journal of
    Physical Oceanography} 5 (4): 572--584.

\item [Wyrtki]K. 1979. Sea level variations: monitoring the breath of
  the Pacific. \textit{EOS Transactions of the American Geophysical
    Union} 60 (3): 25-27.

\item [Wyrtki]K. 1985. Water displacements in the Pacific and the
  genesis of El Ni\~{n}o cycles. \textit{Journal of Geophysical
    Research} 90 (C4): 7129-7132.
	
\item [Xie]P., and P.A. Arkin. 1997. Global precipitation: A 17-year
  monthly analysis based on gauge observations, satellite estimate,
  and numerical model outputs. \textit{ Bulletin of the American
    Meteorological Society} 78 (1): 2539--2558.

\item [Yelland]M., and P.K. Taylor. 1996. Wind stress measurements
  from the open ocean. \textit{Journal of Physical Oceanography} 26
  (4): 541--558.

\item [Yelland]M.J., B.I. Moat, P.K. Taylor, R.W. Pascal, J. Hutchings
  and V.C. Cornell. 1998. Wind stress measurements from the open ocean
  corrected for airflow distortion by the ship. \textit{Journal of
    Physical Oceanography} 28 (7): 1511--1526.

\item [Yu]X., and M.J. McPhaden. 1999. Dynamical analysis of seasonal
  and interannual variability in the equatorial
  Pacific. \textit{Journal of Physical Oceanography} 29 (9):
  2350--2369.

\item [Zahn]R. 1994. Core correlations. \textit{Nature} 371 (6495):
  289--290.

\end{description}
