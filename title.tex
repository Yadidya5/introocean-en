\begin{titlepage} 
\centering
\Huge
\textbf{Introduction \rule{0mm}{55mm}To \\Physical \rule{0mm}{10mm}Oceanography}
\normalsize
\vspace{60mm}                     \\
       Robert H. Stewart          \\
       Department of Oceanography \\ 
       Texas A \& M University    \\
   \rule{0mm}{10mm}Copyright 2008 \\
       September 2008 Edition
\end{titlepage}

%addtocontents{toc}{chapter}{Index}
\tableofcontents

\chapter{Preface}
This book is written for upper-division undergraduates and new graduate students in meteorology, ocean engineering, and oceanography. Because these students have a diverse background, I have emphasized ideas and concepts more than mathematical derivations.

Unlike most books, I am distributing this book for free in digital format via the world-wide web. I am doing this for two reasons:
\begin{enumerate}
\vitem
Textbooks are usually out of date by the time they are published, usually a year or two after the author finishes writing the book. Randol Larson, writing in \textit{Syllabus}, states: ``In my opinion, technology textbooks are a waste of natural resources. They're out of date the moment they are published. Because of their short shelf life, students don't even want to hold on to them''---(Larson, 2002). By publishing in electronic form, I can make revisions every year, keeping the book current.
\vitem
Many students, especially in less-developed countries cannot afford the high cost of textbooks from the developed world. This then is a gift from the US National Aeronautics and Space Administration \textsc{nasa} to the students of the world.
\end{enumerate} 

\section*{Acknowledgements}
I have taught from the book for several years, and I thank the many students in my classes and throughout the world who have pointed out poorly written sections, ambiguous text, conflicting notation, and other errors.  I also thank Professor Fred Schlemmer at Texas A\&M Galveston who, after using the book for his classes, has provided extensive comments about the material.

I also wish to thank many colleagues for providing figures, comments, and helpful information. I especially wish to thank Aanderaa Instruments, Bill Allison, Kevin Bartlett, James Berger, Gerben de Boer, Daniel Bourgault,  Don Chambers, Greg Crawford, Thierry De Mees, Richard Eanes, Peter Etnoyer, Tal Ezer, Gregg Foti, Nevin S. Fu\v{c}kar, Luiz Alexandre de Araujo Guerra, Hazel Jenkins, Andrew Kiss, Jody Klymak, Judith Lean, Christian LeProvost, Brooks Martner, Nikolai Maximenko, Kevin McKone, Mike McPhaden, Thierry De Mees, Pim van Meurs, Gary Mitchum, Joe Murtagh, Peter Niiler, Nuno Nunes, Ismael N\'{u}\~{n}ez-Riboni, Alex Orsi, Kym Perkin, Mark Powell, Richard Ray, Joachim Ribbe, Will Sager, David Sandwell, Sea-Bird Electronics, Achim Stoessel, David Stooksbury, Tom Whitworth, Carl Wunsch and many others.

Of course, I accept responsibility for all mistakes in the book. Please send me your comments and suggestions for improvement.

Figures in the book came from many sources. I particularly wish to thank Link Ji for many global maps, and colleagues at the University of Texas Center for Space Research. Don Johnson redrew many figures and turned sketches into figures. Trey Morris tagged the words used in the index.

I especially thank \textsc{nasa}'s Jet Propulsion Laboratory and the Topex/Poseidon and Jason Projects for their support of the book through contracts 960887 and 1205046.

Cover photograph of the resort island of Kurumba in North Male Atoll in the Maldives was taken by Jagdish Agara (copyright Corbis). Cover design is by Don Johnson.

The book was produced in \LaTeXe\ using TeXShop 2.14 on an Intel iMac computer running OS-X 10.4.11. I especially wish to thank Gerben Wierda for his very useful i-Installer package that made it all possible, Richard Koch, Dirk Olmes and many others for writing the TeXShop software package, and Andrew Kiss at the University of New South Wales in Canberra Australia for help in using the hyperref package. The \LaTeXe\  software is a pleasure to use. All figures were drawn in Adobe Illustrator.
